\documentclass{article}
\usepackage[T1,T2A]{fontenc}
\usepackage[utf8]{inputenc}
\usepackage[english,russian]{babel}
\usepackage{amsmath}
\usepackage{amsfonts}
\usepackage{amssymb}
\usepackage{makeidx}
\usepackage{setspace,amsmath}
\newcommand\tab[1][1cm]{\hspace*{#1}}
\begin{document}
\begin{center}
\Large\bf Простая задача на работу с BMP-изображениями в Python
\end{center}

{\large\bf Задание:}

Формат вызова программы должен быть следующим\\
./prog N InputFile OutputFile

Программа должна загрузить изображение из графического файла InputFile, создать вокруг введенного изображения черную рамку ширины N (ширина и высота изображения, при этом, увеличиваются на 2N).

{\large\bf Решение:}

Установим сначала библиотеку: Pillow. Подключим также модуль Sys для работы с аргументами командной строки.

{\usefont{T2A}{cmss}{m}{n}
import sys

from PIL import Image, ImageDraw}

Изображение для нас - это список, индексируемый парой чисел (кортежем). Откроем его.

{\usefont{T2A}{cmss}{m}{n}
im=Image.open(sys.argv[2])

x,y=im.size

px=im.load()}

Создадим новый список длиной и шириной на 2N больше исходного избражения. Каждый элемент этого списка - тройка чисел - интенсивность красного, зеленого и синего цветов.
Зададим условие копирования пикселя из старого изображения в новое: пиксель старого избражения (x,y) -> (x+N, y+N) пиксель нового.

{\usefont{T2A}{cmss}{m}{n}
\#получили число пикселей в рамке

n=int(sys.argv[1])

\#создали новое изображение

im2=Image.new("RGB", (x+2*n, y+2*n), (0, 0, 0))

px2=im2.load()

for i in range(x+2*n):

\tab[1cm]	for j in range(y+2*n):

\tab[2cm]		if i>n and i<x+n and j>n and j<y+n:

\tab[3cm]			r1,g1,b1=px[i-n,j-n]

\tab[3cm]			px2[i,j]=r1,g1,b1

\tab[2cm]		else:

\tab[3cm]			px2[i,j]=0,0,0}

Остальные элементы заполним черным цветом, то есть тремя нулями.

Сохраним новое изображение в OutputFile.

{\usefont{T2A}{cmss}{m}{n}
im2.save(sys.argv[3])}

Программа вызывается из командной строки следующим образом:

{\usefont{T2A}{cmss}{m}{n}
python3 ramka.py 50 pic.bmp result.bmp}

\newpage
{\large\bf Код полностью:}

{\usefont{T2A}{cmss}{m}{n}
import sys

from PIL import Image, ImageDraw

im=Image.open(sys.argv[2])

x,y=im.size

px=im.load()

n=int(sys.argv[1])

im2=Image.new("RGB", (x+2*n, y+2*n), (0, 0, 0))

px2=im2.load()

for i in range(x+2*n):

\tab[1cm]	for j in range(y+2*n):

\tab[2cm]		if i>n and i<x+n and j>n and j<y+n:

\tab[3cm]			r1,g1,b1=px[i-n,j-n]

\tab[3cm]			px2[i,j]=r1,g1,b1

\tab[2cm]		else:

\tab[3cm]			px2[i,j]=0,0,0

im2.save(sys.argv[3])
}

\end{document}