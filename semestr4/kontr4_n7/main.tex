\documentclass{article}
\usepackage[T1,T2A]{fontenc}
\usepackage[utf8]{inputenc}
\usepackage[english,russian]{babel}
\usepackage{amsmath}
\usepackage{amsfonts}
\usepackage{amssymb}
\usepackage{makeidx}
\usepackage{setspace,amsmath}
\newcommand\tab[1][1cm]{\hspace*{#1}}
\begin{document}
\begin{center}
{\large\bf Отчёт по работе с BMP-изображениями в Python}
\end{center}

\textit{Задача:} Программа должна загрузить изображения из графических файлов InputFile1, InputFile2, обнулить значения всех пикселов в первом изображении, для которых соответствующий пиксел во втором изображении нулевой и вывести получившееся изображение в графический файл OutputFile. Под нулевыми пикселами подразумеваются пикселы, для которых все три компоненты равны 0.

 
\textit{Решение:}  Для решения задачи булем использовать библиотеку PIL:

\textsf{from PIL import Image, ImageDraw }

А также будем использовать библиотеку sys для загрузки файлов из командной строки:

\textsf{import sys}

Загружаем изображения, ищем их размеры, чтобы случайно не выйти за границы при перекрашивании пикселов. Для этого финальное изображение будет минимальным по ширине и длине.

{\usefont{T2A}{cmss}{m}{n}

image = Image.open(sys.argv[1]) 

image2 = Image.open(sys.argv[2]) 

draw = ImageDraw.Draw(image)  

width = image.size[0]  

height = image.size[1]  

width2 = image2.size[0] 

height2 = image2.size[1] 

w=min(width, width2)

h=min(height, height2)

pix = image.load() 

pix2 = image2.load() }


Далее начинаем работать с этими изображениями. Наша цель - найти все нулевые пикселы второго изображения. Будем перебирать их с помощью двойного цикла совместно с пикселами первого изображения
 
{\usefont{T2A}{cmss}{m}{n}

for i in range(w):

\tab[0.5cm] for j in range(h):

\tab[1cm] a = pix2[i, j][0] 

\tab[1cm] a = pix[i, j][0]

\tab[1cm] b = pix[i, j][1]

\tab[1cm] c = pix[i, j][2]

\tab[1cm] a1 = pix2[i, j][0]

\tab[1cm] b1 = pix2[i, j][1]

\tab[1cm] c1 = pix2[i, j][2] }

В том же цикле тут же будем менять цвет пиксела первого изображения на цвет пиксела второго, если условие (0,0,0) для цвета соблюдено. Для этого будем заново рисовать на нашем первом изображении с помощью команды $point$.

\newpage

{\usefont{T2A}{cmss}{m}{n}
if a1==0 and b1==0 and c1==0:

\tab[0.5cm] draw.point((i, j), (a1, b1, c1))
	    
else:
		     
\tab[0.5cm] draw.point((i, j), (a, b, c))}

Сохраним изображение и очистим память от элемента $draw$:	
	
{\usefont{T2A}{cmss}{m}{n}
image.save(sys.argv[3])

del draw
 }
  
\end{document}